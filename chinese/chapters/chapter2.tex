\chapter{基本原理:解决泊松方程}
\label{ch:fundamentals}

\begin{summary}
本章的目标是展示泊松方程如何
所有 PDE 中最基本的都可以用几行快速解决
的 FEniCS 代码。 我们介绍最多
基本 FEniCS 对象:
\texttt{Mesh}, \texttt{Function}, \texttt{FunctionSpace}, \texttt{TrialFunction}, \texttt{TestFunction}。
了解如何编写基本的 PDE 求解器,
包括如何制定数学变分问题,
应用边界条件,调用 FEniCS 求解器和绘图
解决方案。
\end{summary}


\section{数学问题的制定}
\label{ftut:poisson1:bvp}
\index{Poisson's equation}

许多关于编程语言的书籍都以HelloWorld开头
程序。 读者好奇知道基本的任务是什么
用语言表达,并将文字打印到屏幕上即可
这样的一个任务。 在有限元方法的世界,PDE,
最基本的任务是解决泊松方程。 我们的
因此对应于古典HelloWorld程序
解决以下边界值问题:

\begin{alignat}{2}
- \nabla^2 u(\x) &= f(\x),\quad &&\x\mbox{ in } \Omega,
\label{ftut:poisson1}\\
u(\x) &= \ub(\x),\quad &&\x\mbox{ on } \partial \Omega\tp \label{ch:poisson0:bc}
\end{alignat}

这里,$u = u(\x)$是未知函数,$f = f(\x)$是
规定的函数,$\nabla^2$是Laplace算子
(通常写为$\Delta$),$\Omega$是空间域,而
$\partial \Omega$是$\Omega$的边界。 Poisson问题,
包括PDE $-\nabla^2u = f$和边界条件
$u = \ub$ on $\partial \Omega$,是边界值的一个例子
问题,必须在之前精确地陈述
使用FEniCS开始解决它是有意义的。

在具有坐标$x$和$y$的两个空间维度中,我们可以写出来
Poisson方程为

\begin{equation}
- {\partial^2 u\over\partial x^2} -
{\partial^2 u\over\partial y^2} = f(x,y)\tp
\end{equation}

未知的$u$现在是两个变量$u = u(x,y)$的函数
在二维域$\Omega$。

Poisson方程出现在许多物理环境中,包括
热传导,静电,物质扩散,扭转
弹性棒,非粘性流体流和水波。 而且,
方程出现在数值分割策略中更为复杂
PDE系统,特别是Navier-Stokes方程。

解决边界值问题,如Poisson方程
FEniCS包括以下步骤:

\begin{enumerate}
\item 识别计算域($\Omega$),PDE,它 边界条件和源项($f$)。
\item 将PDE重新定义为有限元变分问题。
\item 编写一个定义计算域的Python程序,变分问题,边界条件和来源
  条款,使用相应的FEniCS抽象。
\item 调用FEniCS来解决边界值问题,并且可选地,扩展程序
计算衍生量,如通量和平均值,以及 可视化结果。
\end{enumerate}

\noindent
现在我们将详细介绍第2--4步。 的主要特点
FEniCS是步骤3和4导致相当短的代码,而a
大多数其他PDE软件框架中的类似程序都需要
更多的代码和技术上困难的编程。

\begin{notice}[什么使FEniCS有吸引力?]
虽然许多软件框架都非常优雅
HelloWorld 的例子
Poisson方程式,FEniCS是我们知道的唯一框架
代码保持紧凑和美观,非常接近数学
即使是数学和算法的复杂性
从笔记本电脑转移到高性能时会增加
计算服务器(集群)。
\end{notice}

\subsection{有限元变分法}
\label{ch:poisson0:varform}
\index{variational formulation}

FEniCS是基于有限元法,它是一般的和
高效数学机械的数值解
PDE。 有限元方法的出发点是PDE
以变体形式表达。 不熟悉的读者
变数问题将会简要介绍一下这个话题
在本教程中,但阅读有限元上的正确书
鼓励方法。 经验表明,您可以使用
FEniCS作为解决PDE的工具,即使没有深入的了解
有限元法,只要你有人帮你
将PDE作为变分问题。

\index{test function}
\index{trial function}

将PDE转化为变分问题的基本方法是
将PDE乘以函数$v$,整合得到的等式
通过域$\Omega$,并按部分条款执行整合
与二阶导数。 函数$v$乘以
PDE称为测试功能。 未知函数$u$为
近似被称为试验函数。 试用版和
测试功能也用于FEniCS程序。 试用和测试
功能属于指定的所谓功能空间
功能的属性。

\index{integration by parts}

在这种情况下,我们先乘以Poisson方程
通过测试函数$v$并集成$\Omega$:

\begin{equation}
\label{ch:poisson0:multbyv}
-\int_\Omega (\nabla^2 u)v \dx = \int_\Omega fv \dx\tp
\end{equation}
我们这里让$\dx$表示用于集成的差分元素
域$\Omega$。 我们稍后会让$\ds$表示差分
在$\Omega$的边界上整合的元素。

当我们得出变分公式时,一个常见的规则就是我们尝试
保持$u$和$v$的衍生工具的顺序尽可能小
可能。 在这里,我们有一个$u$的二阶空间导数,
这可以转换为$u$和$v$的一阶导数
应用零件整合技术。 公式
读

\begin{equation}
\label{ch:poisson0:eqbyparts}
 -\int_\Omega (\nabla^2 u)v \dx
= \int_\Omega\nabla u\cdot\nabla v \dx - \int_{\partial\Omega}{\partial u\over
\partial n}v \ds,
\end{equation}

其中$\frac{\partial u}{\partial n} = \nabla u \cdot n$是
$u$的衍生物向外向正方向$n$
边界。

变分配方的另一个特点是
测试函数$v$需要在部分消失
解决方案$u$的边界已知。
在现在
问题,这意味着$v = 0$在整个边界$\partial \Omega$。
第二个术语在右边
(\ref{ch:poisson0:eqbyparts})因此消失。 从
(\ref{ch:poisson0:multbyv})和(\ref{ch:poisson0:eqbyparts})
遵循

\begin{equation}
\int_\Omega\nabla u\cdot\nabla v \dx = \int_\Omega fv \dx\tp
\label{ch:poisson0:weak1}
\end{equation}

如果我们要求该方程适用于所有测试函数$v$
一些合适的空间$\hat V$,所谓的测试空间,我们获得一个
明确的数学问题,唯一地决定了
解决方案$u$在于一些可能不同的功能空间
$V$,所谓的试验空间。 我们参考
(\ref{ch:poisson0:weak1})作为弱形式或变体形式
原边界值问题
(\ref{ftut:poisson1})--(\ref{ch:poisson0:bc})。

适当的陈述
我们的变分问题现在如下:
找到$u \in V$这样

\begin{equation} \label{ch:poisson0:var}
  \int_{\Omega} \nabla u \cdot \nabla v \dx =
  \int_{\Omega} fv \dx
  \quad \forall v \in \hat{V}\tp
\end{equation}

现在试用和测试空间$V$和$\hat V$
问题定义为

\begin{align*}
     V      &= \{v \in H^1(\Omega) : v = \ub \mbox{ on } \partial\Omega\}, \\
    \hat{V} &= \{v \in H^1(\Omega) : v = 0 \mbox{ on } \partial\Omega\}\tp
\end{align*}
简而言之,$H^1(\Omega)$是数学上众所周知的Sobolev空间
包含函数$v$,使$v^2$和$|\nabla v|^2$有
有限积分超过$\Omega$(基本上意味着函数
是连续的)。 底层PDE的解决方案必须在于
函数空间中的衍生物也是连续的,但是
Sobolev空间$H^1(\Omega)$允许不连续的函数
衍生物。 $u$的连续性要求较弱
变量语句(\ref{ch:poisson0:var}),作为结果
整合部分,具有很大的实际后果
构造有限元函数空间。 特别是它
允许使用分段多项式函数空间; 即功能
通过简单地将多项式函数拼接在一起构成的空间
域,如间隔,三角形或四面体。

变量问题(\ref{ch:poisson0:var})是连续的
问题:它定义了无穷维的解$u$
功能空间$V$。 Poisson方程的有限元法
找出变分问题的近似解
(\ref{ch:poisson0:var})替换无限维函数
空间$V$和$\hat{V}$通过离散(有限维)试验
测试空间$V_h\subset{V}$和$\hat{V}_h\subset\hat{V}$。 离散变分问题如下:
找到$u_h \in V_h \subset V$这样

\begin{equation} \label{ch:poisson0:vard}
  \int_{\Omega} \nabla u_h \cdot \nabla v \dx =
  \int_{\Omega} fv \dx
  \quad \forall v \in \hat{V}_h \subset \hat{V}\tp
\end{equation}

这个变分问题,连同一个合适的定义
函数空间$V_h$和$\hat{V}_h$,唯一地定义我们的近似值
Poisson方程的数值解(\ref{ftut:poisson1})。 注意
边界条件被编码为试验和测试的一部分
空间。 起初,数学框架可能看起来很复杂
一瞥,但好消息是有限元变分
问题(\ref{ch:poisson0:vard})看起来和连续的一样
变分问题(\ref{ch:poisson0:var})和FEniCS可以
自动解决变量问题,如(\ref{ch:poisson0:vard})!

\begin{notice}[我们的意思是符号$u$和$V$]
关于变数问题的数学文献写了$u_h$
解的离散问题和$u$的解决方案
连续问题 获得(几乎)一对一的关系
在一个问题的数学表达与之间
相应的FEniCS程序,我们将下拉$_h$和使用
$u$用于解决离散问题。
我们将使用$\uex$来确定
解决连续问题,如果我们需要明确区分
两者之间。 类似地,我们将让$V$表示离散有限
元素功能空间,我们寻求我们的解决方案。
\end{notice}

\subsection{抽象有限元变分公式}
\label{ch:poisson0:abstrat}
\index{abstract variational formulation}

原来是方便的介绍下列规范
变量问题的符号:找到$u \in V$这样

\begin{equation}
a(u, v) = L(v) \quad \forall v \in \hat{V}.
\end{equation}

对于Poisson方程,我们有:

\begin{align}
a(u, v) &= \int_{\Omega} \nabla u \cdot \nabla v \dx,
\label{ch:poisson0:vard:a}\\
L(v) &= \int_{\Omega} fv \dx\tp  \label{ch:poisson0:vard:L}
\end{align}

从数学文献中,$a(u,v)$被称为双线性
形式和$L(v)$作为线性形式。 我们将在每一个线性问题中
我们解决,确定与未知$u$的条款并收集他们
$a(u,v)$,并且类似地收集只有已知功能的所有术语
$L(V)$。 $a$和$L$的公式可以直接表达
我们的FEniCS计划。

为了解决FEniCS中的线性PDE,如Poisson方程,用户
因此需要执行两个步骤:

\begin{itemize}
\item 通过指定选择有限元空间$V$和$\hat V$
域(网格)和函数空间的类型(多项式)
度和类型)。
\item 将PDE表示为(离散)变分问题:找到$u\in V$
所以$a(u,v) = L(v)$为$v\in \hat{V}$。
\end{itemize}

\noindent
\subsection{选择测试问题}
\label{ch:poisson0:testproblem}

泊松问题 (\ref{ftut:poisson1})--(\ref{ch:poisson0:bc})有
远程功能一般域$\Omega$和一般功能$\ub$
边界条件和右边的$f$。 对于我们的第一个
实现我们将需要为$\Omega$做出具体的选择,
$\ub$和$f$。 构建一个已知的问题是明智的
分析解决方案,使我们可以方便地检查计算
解决方案是正确的。 低阶多项式的解是
主要候选人 标准有限元函数空间的度数
$r$将完全重现度为$r$的多项式。 分段
线性元素($r = 1$)能够精确地再现二次方
均匀分布网格上的多项式。 这个重要的结果可以
用于验证我们的实现。 我们只是制造一些
比如说2D中的二次函数作为确切的解决方案

\index{test problem}

\begin{equation}
\label{ch:poisson0:impl:uex}
\uex(x,y) = 1 +x^2 + 2y^2\tp
\end{equation}

通过将(\ref{ch:poisson0:impl:uex})插入到Poisson方程式中
(\ref{ftut:poisson1}),我们发现$\uex(x,y)$是一个解决方案,以便

\[ f(x,y) = -6,\quad \ub(x,y)=\uex(x,y)=1 + x^2 + 2y^2,\]
只要$\uex$被规定,不管域的形状
边界。 我们在这里选择,为了简单,
该域为单位广场,

\[ \Omega = [0,1]\times [0,1] \tp\]
这个简单但非常强大的构建测试问题的方法是
称为制造解决方案:选择一个简单
表达式为精确解,将其插入等式获得
右边(源项$f$),然后用公式求解
这个右手边使用精确的解决方案作为边界
条件,并尝试重现确切的解决方案。

\index{verification}
\index{exact solution}
\index{method of manufactured solutions}

\begin{notice}[提示:尝试使用精确的数值解决方案验证您的代码!]
测试数值方法实现的常用方法
是比较数字
解决方案具有测试问题的精确解析解
得出结论,如果错误是“足够小”,该程序可以工作。
不幸的是,不可能知道一个大小为$10^{ - 5}$的错误
$20\times 20$ mesh的线性元素是预期的准确度
数值近似或者如果误差也包含a的影响
代码中的错误。我们通常都知道数值误差是它的
\emph{asymptotic properties},例如它与$h^2$成正比
如果$h$是网格中单元格的大小。然后我们比较
使用不同的$h$-values来查看网格的错误,看是否渐近
行为是正确的。这是一个非常强大的验证
技术,并在〜\ref{ch:poisson0:convrates}部分中详细解释。
但是,如果我们有一个测试问题
我们知道应该没有近似误差,我们知道
PDE问题的分析解决方案应该重现
机器精度由程序。这就是为什么我们强调这种
本教程中的测试问题。通常,元素
degree $r$可以重现度为$r$的多项式,所以这样
是构建无数值解的起点
近似误差
\end{notice}

\section{FEniCS实现}
\label{ch:poisson0:impl}

\subsection{完整的程序}
一个用于解决Poisson方程的测试问题的FEniCS程序
在2D中,给定的$\Omega$,$\ub$和$f$的选项可能看起来像
如下:

\begin{bash}
from fenics import *

# Create mesh and define function space
mesh = UnitSquareMesh(8, 8)
V = FunctionSpace(mesh, 'P', 1)

# Define boundary condition
u_D = Expression('1 + x[0]*x[0] + 2*x[1]*x[1]', degree=2)

def boundary(x, on_boundary):
    return on_boundary

bc = DirichletBC(V, u_D, boundary)

# Define variational problem
u = TrialFunction(V)
v = TestFunction(V)
f = Constant(-6.0)
a = dot(grad(u), grad(v))*dx
L = f*v*dx

# Compute solution
u = Function(V)
solve(a == L, u, bc)

# Plot solution and mesh
plot(u)
plot(mesh)

# Save solution to file in VTK format
vtkfile = File('poisson/solution.pvd')
vtkfile << u

# Compute error in L2 norm
error_L2 = errornorm(u_D, u, 'L2')

# Compute maximum error at vertices
vertex_values_u_D = u_D.compute_vertex_values(mesh)
vertex_values_u = u.compute_vertex_values(mesh)
import numpy as np
error_max = np.max(np.abs(vertex_values_u_D - vertex_values_u))

# Print errors
print('error_L2  =', error_L2)
print('error_max =', error_max)

# Hold plot
interactive()
\end{bash}

该示例程序可以在文件中找到{\nolinkurl{ft01_poisson.py}}。
\begin{center}
  \url{https://fenicsproject.org/pub/tutorial/python/vol1/ft01_poisson.py}
\end{center}

\index{ft01\_poisson.py@{\rm\texttt{ft01\_poisson.py}}}

\subsection{运行程序}
\label{ch:poisson0:impl:run}

FEniCS程序必须以纯文本文件的形式提供
文本编辑器,如Atom,Sublime Text,Emacs,Vim等。
有几种方法可以运行Python程序{\nolinkurl{ft01_poisson.py}}:
\begin{center}
  \url{https://fenicsproject.org/pub/tutorial/python/vol1/ft01_poisson.py}
\end{center}

\begin{itemize}
 \item 使用终端窗口。

 \item 使用集成开发环境(IDE),例如Spyder。

 \item 使用Jupyter笔记本。
\end{itemize}

\noindent
\paragraph{终端窗口。}
\index{terminal window}

打开一个终端
窗口,移动到包含程序的目录并键入
以下命令:

\begin{bash}
$ python ft01_poisson.py
\end{bash}

请注意,此命令必须在支持FEniCS的终端中运行。 对于
FEniCS Docker容器的用户,这意味着您必须键入
启动FEniCS会话后使用此命令
\texttt{fenicsproject run}或\texttt{fenicsproject start}。

运行上述命令时,FEniCS将运行程序进行计算
近似解$u$。 大概的解决方案$u$将是
与确切的解决方案$\uex = \ub$相比,$L^2$和
将打印最大规范。 既然我们知道我们的大概
解决方案应该在机器内重现精确的解决方案
精度,这个错误应该是小的,有的是顺序的
$10^{-15}$。 如果您的FEniCS安装中启用了绘图,
那么一个简单的解决方案的窗口将显示为
在图〜\ref{fig:poisson_plot}中。

\paragraph{Spyder。}
\index{Spyder}

许多人喜欢在一个集成的开发环境中工作
提供编程编辑器,执行代码的窗口,
用于检查对象的窗口等。只需打开文件
{\nolinkurl{ft01_poisson.py}}
\begin{center}
  \url{https://fenicsproject.org/pub/tutorial/python/vol1/ft01_poisson.py}
\end{center}
并按播放按钮运行它。 我们参考Spyder教程
了解更多关于在Spyder环境中工作的信息。 Spyder是
强烈推荐,如果你习惯在图形工作
MATLAB环境。

\paragraph{Jupyter notebooks。}
\index{Jupyter}

笔记本电脑可以将文本和可执行代码混合在一起
文档,但您也可以使用它来在网络中运行程序
浏览器。从终端窗口运行命令\texttt{jupyter notebook}
在GUI的右上角找到\textbf{New}下拉菜单,
在Python 2或3中选择一个新的notebook,写入\verb!%load ft01_poisson.py!在这个notebook的空白单元格中,然后按
Shift + Enter执行单元格。然后,将文件\nolinkurl{ft01_poisson.py}
\begin{center}
  \url{https://fenicsproject.org/pub/tutorial/python/vol1/ft01_poisson.py}
\end{center}
加载到
notebook。重新执行cell(Shift + Enter)运行程序。您
可以将整个程序划分成几个单元格进行检查
中间结果:将光标放在要分割的位置
cell并选择\textbf{Edit - Split Cell}。对于FEniCS Docker的用户
图像,运行\texttt{fenicsproject notebook}命令并按照
说明。要启用绘图,请确保运行该命令
\verb!%matplotlib inline!在notebook里面。

\section{解剖方案}
\label{ch:poisson0:impl:dissect}
我们现在将详细剖析我们的FEniCS计划。 列出的FEniCS
程序定义有限元网格,有限元函数空间
$V$在此网格上,边界条件为$u$(函数$\ub$),
和双线性和线性形式$a(u,v)$和$L(v)$。 此后,
解决方案$u$被计算。 在程序结束时,我们比较了
数值和确切的解决方案。 我们也使用该图来绘制解决方案
\texttt{plot}命令并将解决方案保存到外部文件
后期处理。

\subsection{重要的第一行}
程序的第一行,
\begin{bash}
from fenics import *
\end{bash}
导入关键类\texttt{UnitSquareMesh},\texttt{FunctionSpace},\texttt{Function},
等等,从FEniCS图书馆。 所有FEniCS程序
通过有限元法解决PDE通常从这开始
线。

\index{mesh}
\index{Mesh@{\rm\texttt{Mesh}}}

\subsection{生成简单的网格}
该声明
\begin{bash}
mesh = UnitSquareMesh(8, 8)
\end{bash}
在单位平方上定义均匀的有限元网格
$[0,1]\times [0,1]$。 网格由\emph{cells}组成,其中2D是三角形
有直边。 参数8和8指定正方形
应分为$8\times 8$矩形,每个分为一对
三角形。 三角形(cells)的总数因此变为
128。 网格中的顶点总数为$9\cdot 9=81$。
在后面的章节中,您将学习如何生成更复杂的网格。

\index{FunctionSpace@{\rm\texttt{FunctionSpace}}}
\index{finite element space}
\index{Lagrange finite element}
\index{P1 element}

\subsection{定义有限元函数空间}
\index{FunctionSpace@{\rm\texttt{FunctionSpace}}}
\index{function space}

一旦创建了网格,就可以创建一个有限元的功能空间
\texttt{V}:

\begin{bash}
V = FunctionSpace(mesh, 'P', 1)
\end{bash}

第二个参数\texttt{'P'}指定元素的类型。 的类型
这里的元素是$\mathsf{P}$,意味着标准的Lagrange系列
元素。 您也可以使用\texttt{'Lagrange'}来指定此类型
元件。 FEniCS支持所有单体元素族和符号
定义在\footnote{\texttt{https://www.femtable.org}}{有限元素周期表}\cite{ArnoldLogg2014}。
\begin{center}
  \url{https://www.femtable.org}
\end{center}

\index{Periodic Table of the Finite Elements}

第三个参数\texttt{1}指定有限元的程度。 在
这种情况下,标准的$\mathsf{P}_1$ linear xxzxLagrange元素,其中
是三角形,节点在三个顶点。 一些有限元
从业者将此元素称为“线性三角形”。该
计算的解决方案$u$将在元素和线性上是连续的
每个元素内的$x$和$y$变化。 高阶多项式
通过增加每个单元的近似值
\texttt{FunctionSpace}的第三个参数,然后生成函数
类型为$\mathsf{P}_2$,$\mathsf{P}_3$的空格等等。更改
\texttt{'DP'}的第二个参数创建一个函数空间
不连续的Galerkin方法。

\index{TestFunction@{\rm\texttt{TestFunction}}} \index{TrialFunction@{\rm\texttt{TrialFunction}}}
\index{DirichletBC@{\rm\texttt{DirichletBC}}}
\index{Dirichlet boundary conditions}
